% This is a LaTeX template for ICPEAC 2021 abstract
% Acknowledgements: thanks to Dragan Nicolic
% Modifications by H Bachau, F Penent and R Brédy for ICPEAC 2019
% (Tiny) modifications by T Jahnke for ViCPEAC 2021
% Last update: 17/02/2021

%%%%%%%%%%%%%%%%%%%%%%%%%%%%%%%%%%%%%%%%%%%%
%					USER'S INPUT IS EXPECTED below line 135
%%%%%%%%%%%%%%%%%%%%%%%%%%%%%%%%%%%%%%%%%%%%

\documentclass[a4paper,11pt,draft]{article}

%%%%%%% PACKAGES %%%%%%%
\usepackage[a4paper,bindingoffset=0truemm,%
            left=20truemm,right=20truemm,top=30truemm,bottom=29truemm,
            footskip=0truemm]{geometry} % (do not change dimensions)
\usepackage{multicol}
\usepackage{graphicx}
\usepackage{setspace}
\usepackage[hang,flushmargin]{footmisc} %left align the footnote - no indent
\usepackage[
   colorlinks,
   linkcolor=blue,
   urlcolor=blue,
   citecolor=blue,
%   dvipdfm           % Uncomment if using dvipdfm to convert dvi into pdf
   ]{hyperref}
%\usepackage{layout}

%%%%%%%%%%%%%%%%%%%% MACROS (do not change) %%%%%%%%%%%%%%%%%%%%

\renewcommand\refname{\normalsize References }

\renewcommand\baselinestretch{1}
\pagestyle{empty}

\newcommand{\abstracttitle}[1]{
\setstretch{1.15}	%increase the space between the two lines of the title
 \begin{center}{\Large {\bf #1}}\end{center}
\setstretch{1.0}
}

\newcommand{\authors}[1]{
%\vspace*{-10pt}
 \begin{center}{\bf #1} \end{center}
 \setstretch{1.0}
%\vspace*{-10pt}
}

\newcommand{\addresses}[1]{
\setstretch{0.95} %decrease the space between the affiliation
 \begin{center}{\small #1} \end{center}
\setstretch{1.0}
}

\newcommand{\synopsis}[1]{
 \begin{center}
%  \setstretch{0.85}
 \begin{minipage}[t]{16cm}
   \setstretch{0.85}
   {\footnotesize {\bf Synopsis} #1 }	
 \end{minipage}
 \end{center}
 \setstretch{1.0}
}

\newcommand{\abstracttext}[1]{
% \vspace*{-0.3cm}
\vspace*{0.10cm}
 \columnsep0.75cm
 \begin{multicols}{2} #1 \end{multicols}
% \vspace{\fill}
}

\newcommand{\capt}[2]{
 \vspace*{-0.3cm}
 \begin{center}
 \begin{minipage}[t]{7.8cm} {\small {\bf Figure~#1}.~#2}
  \end{minipage}
 \end{center}
% \vspace*{0.3cm}
}

\newcommand{\captTab}[2]{
% \vspace*{-0.3cm}
 \vspace*{-0.1cm}
 \begin{center}
 \begin{minipage}[t]{7.8cm} {\small {\bf Table~#1}.~ #2}
 \end{minipage}
 \end{center}
 \vspace*{0.1cm}
}

\let\OLDthebibliography\thebibliography
\renewcommand\thebibliography[1]{
  \OLDthebibliography{#1}
  \setlength{\parskip}{0pt}
  \setlength{\itemsep}{0pt plus 0.3ex}
}

%%\renewcommand{\thefootnote}{\alph{footnote}}
\renewcommand{\thefootnote}{\fnsymbol{footnote}}
\newcommand{\writeto}[1]{
 \hspace*{-2.5mm} \footnote{\small E-mail: \href{mailto:#1}{#1}} %add \small= 10pt here
  \hspace*{-3.0mm} %decrease from -1.5 to -3.0mm
}

%%%%%%%%%%%%%%%%%%%% MACROS to isert the  figure %%%%%%%%%%%%%%%%%%%%
% you may change the scale factor to adjust the size of the figure
%you may crop your figure by adding the option [trim=left bottom right top, clip]
%  and adjust the left bottom right top value as for example
%\includegraphics*[trim = 15mm 10mm 20mm 15mm, clip,width=7.8cm,angle=#1,scale=0.92]{#2}

% Use \picturelportrait{0} when you want to include a portrait figure!
\newcommand{\pictureportrait}[2]{
 \vspace*{0.5cm}
 \centerline{
  \includegraphics*[width=7.8cm,angle=#1,scale=1.0]{#2}
%%% Here the figure scale is "scale=1.0", it can be modified if
%%% necessary (e.g., "scale=0.7" to reduce the figure)
%%%  \includegraphics*[width=7.8cm,height=9cm,angle=#1]{#2}
 }
}
% Use \picturelandscape{0} when you want to include a landscape figure!
\newcommand{\picturelandscape}[2]{
 \vspace*{0.5cm}
 \centerline{
  \includegraphics*[width=7.8cm,angle=#1,scale=0.90]{#2}
%%% Here the figure scale is "scale=0.92", it can be modified if
%%% necessary (e.g., "scale=0.7" to reduce the figure)
%%%  \includegraphics*[width=7.8cm,height=5.5cm,angle=#1]{#2}
%%%\includegraphics*[trim = 15mm 10mm 20mm 15mm, clip,width=7.8cm,angle=#1,scale=0.92]{#2}
 }
}

%%%%%%%%%%%%%%%%% USER'S INPUT IS EXPECTED BELOW THIS LINE %%%%%%%%%%%%%%%%%

\begin{document}
\abstracttitle{
Instructions for preparation of abstracts for ViCPEAC 2021

second line of title if needed but restrict the title to two lines
}%end \abstracttitle

\authors{
A Einstein$^{1}$\writeto{corresponding.author1@abcdef.edu},
W E Pauli$^{2}$\writeto{corresponding.author2@ghijkl.edu}
and E Schr\"odinger$^{1,2}$

Use only initial letters for first name, no full stops; use superscript number\textmd{$^{1,2\dots}$} for affiliation (10 pt)
and symbol\textmd{$^{*,\dagger}$} for corresponding author(s) (2 max) in footnote (10 pt) %(maximum 2 emails)
} %end \authors

\addresses{
$^1$Affiliation with laboratory and/or University name, city, zip code,  country

$^2$Affiliation with laboratory and/or University name, city, zip code,  country
}%end \addresses

\synopsis{\hspace*{2mm}
 %Please replace the text with your synopsis.
The synopsis should be explicit and concise with a maximum length of 600 characters including spaces. Use single-spaced lines with 10 pt Roman font.
} %end \synopsis

\abstracttext{
This is a sample abstract for ViCPEAC 2021.

1.
Abstracts should be written in English.
The abstract should fit on one A4 page without the page number.
It should be submitted in PDF format together with the source files (text and figure) through the conference web site.
Files larger than 1 Mb will be rejected.

2. It is your responsibility to check that the abstract is correctly formatted and readable.
Incorrectly formatted abstracts will be rejected.
Do not modify the pre-defined margins.

3. Use 11 pt Roman font for the text or a similar font such as Times New Roman throughout the abstract.
Font sizes are listed in table~1.

\begin{center}
\captTab{1}{Font sizes to be used in the abstract.}
\begin{tabular}[b]{lccc}
\hline
   & Title & Authors-               & Affiliations-  \\
   &       & body                  & synopsis- \\
   &       &                       & references  \\
\hline
Size & 14~pt & 11~pt & 10~pt  \\
\hline
\end{tabular}
\end{center}

4. The title should be typed in 14 pt boldface font and centered.

5. Leave an empty line (simple, 9 pt) after the title line(s), after the author(s) list and after the affiliation(s) list.

6. Leave an empty line of 18 pt after the synopsis.
Type the bodytext in two-column format with 0.75~cm column separation.
The text should be single-spaced and the first line of each paragraph should be indented by 0.5~cm.

7. Only one figure file, which may display subfigures, is allowed.
It will be shown in colour in the electronic version of the abstract.
Give to the figure a separate caption (10 pt) and, like a table caption (10 pt), it should be concise.
If you submit a colour figure it is your responsibility to ensure that it is easy to read and to be understood even in grayscale mode.
When including the figure, pay close attention to not produce an oversized final PDF file.

\picturelandscape{0}{ViCPEAC-logo.eps}
%\picturelandscape{0}{ViCPEAC-logo.pdf}
%\picturelandscape{0}{ViCPEAC-logo.png}
%\picturelandscape{0}{ViCPEAC-logo.jpg}
\capt{1}{A typical conference logo employing a soothing color scheme.}

8. References in the text should be given by numbers in square brackets~\cite{mar01,ref2} ("Vancouver system").
The list of references (10~pt) follows the last line of the text with one extra line inserted.
Follow the style of JPCS (as Ref.~\cite{mar01} below).



\begingroup
\vspace{-0.25cm}
\small
\begin{thebibliography}{1}
\vspace{-0.25cm}

%insert your references here using \bibitem
\bibitem{mar01} Martin B {\em et al} 2001 {\em J. Phys.: Conf. Ser.}
\href{http://iopscience.iop.org/article/10.1088/1742-6596/875/3/022021/pdf}
{{\bf 123} 012021}

\bibitem{ref2} Reference 2 and so on

\end{thebibliography}
\endgroup
}%\abstracttext

% end of the body
\end{document}
