\documentclass[12pt]{article}
%\documentclass[10pt, landscape, twocolumn]{article}

\usepackage{setspace}
\onehalfspacing
%\usepackage{russ}
\usepackage[T2A]{fontenc}
\usepackage[utf8]{inputenc}
\usepackage{amsmath}

\usepackage[lastpage,user]{zref}


\usepackage{geometry, graphics, graphicx}

\geometry
{
    a4paper,
    total={210mm,297mm},
    left=20mm,
    right=20mm,
    top=25mm,
    bottom=20mm,
}

\usepackage{hyperref}

\hypersetup{
%    bookmarks=true,         % show bookmarks bar?
%    unicode=false,          % non-Latin characters in Acrobat’s bookmarks
%    pdftoolbar=true,        % show Acrobat’s toolbar?
%    pdfmenubar=true,        % show Acrobat’s menu?
%    pdffitwindow=false,     % window fit to page when opened
%    pdfstartview={FitH},    % fits the width of the page to the window
%    pdftitle={My title},    % title
%    pdfauthor={Author},     % author
%    pdfsubject={Subject},   % subject of the document
%    pdfcreator={Creator},   % creator of the document
%    pdfproducer={Producer}, % producer of the document
%    pdfkeywords={keyword1, key2, key3}, % list of keywords
%    pdfnewwindow=true,      % links in new PDF window
    colorlinks=false,       % false: boxed links; true: colored links
    linkcolor=red,          % color of internal links (change box color with linkbordercolor)
    citecolor=green,        % color of links to bibliography
    filecolor=magenta,      % color of file links
    urlcolor=cyan           % color of external links
}
% ----------------------------------------
% ------------Колонтитулы-----------------
\usepackage{fancyhdr}
\pagestyle{fancy}

\fancyhead{}
\lhead{\it Задание 1}
\chead{вариант №\,0}
\rhead{\textbf{Выполнили:} ФИО}
\lfoot{\scriptsize\textbf{UPD.1:}~\emph{21 апреля, 2018}}
\cfoot{}
\rfoot{\thepage /\zpageref{LastPage}}


\usepackage{amsopn}
\DeclareMathOperator{\B}{B}

\usepackage{hyperref}


\usepackage{tikz}
\usepackage{background}
\usepackage{tikzpagenodes}
%\usepackage{lmodern}
\usepackage{indentfirst}

\backgroundsetup%
{   angle=0,
    opacity=2,
    scale=1,
    contents=%
    {   \begin{tikzpicture}[remember picture,scale=3]
            \fontsize{100}{120}\selectfont
            \node[text=gray!50,rotate=90, above=1cm] 	at (current page text area.west) {\textbf{ФН--1}};
            \node[text=gray!50,rotate=-90, above=1cm] 	at (current page text area.east) {\textbf{ФН--1}};          
        \end{tikzpicture}
    }
}
% ----------------------------------------

\newcommand{\ul}{\underline}

\usepackage{amsopn}
\DeclareMathOperator{\up}{up}
\DeclareMathOperator{\h}{h}
\DeclareMathOperator{\fup}{fup}
%\DeclareMathOperator{\ch}{ch}
\DeclareMathOperator{\g}{g}


\begin {document}
\subsection*{\centerline{Вычисление атомарных функций}}
\subsubsection*{Постановка задачи}
\noindent
Вычислить атомарную функцию 
$$
y(x) = \up(x).
$$

\subsubsection*{Решение}
\noindent
\textbf{1.~}Графики функций $\up(x)$ и $\up'(x)$ представлены на рис.\,\ref{fig:01}.
\begin{figure}[h!]
	\center
	\includegraphics[scale=1.15]{up_plot}
	\caption{График функции $\up(x)$ и $\up'(x)$}
	\label{fig:01}
\end{figure}

%\newpage
\noindent
\textbf{2.~}Графики аналитической (\emph{зелёная} кривая) и численной (\emph{красная} кривая) 
производных $\up'(x)$ представлены на рис.\,\ref{fig:02}.
\begin{figure}[h!]
	\center
	\includegraphics[scale=1.15]{up_plot_deriv}
	\caption{График функции $\up(x)$ и $\up'(x)$}
	\label{fig:02}
\end{figure}

\noindent
Введём обозначение
\begin{equation}\label{eq:01}
r(x):= \up'(x) - 2(\up(2x+1)-\up(2x-1)).
\end{equation}
Слагаемое $\up'(x)$ в правой части равенства \eqref{eq:01} вычислено приближённо.
График модуля невязки $r(x)$ представлен на рис.\,\ref{fig:03}.

\begin{figure}[h!]
	\center
	\includegraphics[scale=1.15]{up_plot_deriv_err}
	\caption{График функции $|r(x)|$}
	\label{fig:03}
\end{figure}

 
\end {document}
